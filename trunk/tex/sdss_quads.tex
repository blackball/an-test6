\documentclass[12pt,preprint]{aastex}

\newcommand{\latin}[1]{\textit{#1}}
\newcommand{\ie}{\latin{i.e.}}
\newcommand{\eg}{\latin{e.g.}}
\newcommand{\cf}{\latin{cf.}}
\newcommand{\etc}{\latin{etc.}}
\newcommand{\etal}{\latin{et~al.}}

\newcommand{\usnob}{USNO-B1.0}

\newcommand{\starlabel}[1]{\mathsf{#1}}
\newcommand{\AAA}{\starlabel{A}}
\newcommand{\BBB}{\starlabel{B}}
\newcommand{\CCC}{\starlabel{C}}
\newcommand{\DDD}{\starlabel{D}}
\newcommand{\NABCD}{N_{\AAA\BBB\CCC\DDD}}
\newcommand{\ENABCD}{E(\NABCD)}
\newcommand{\thetaAB}{\theta_{\AAA\BBB}}
\newcommand{\thetamin}{\theta_\mathrm{min}}
\newcommand{\thetamax}{\theta_\mathrm{max}}

\begin{document}

\title{
  Automated Astrometry I:\\
  Blind determination of image pointing, rotation, and scale
}
\author{
  [tba]
}

\begin{abstract}
We report on the first stage of an ambitious project in the automated
determination of image world coordinate systems (WCS), \ie, automated
determination of the precise relationship between $(x,y)$ positions in
an image and (RA,Dec) positions on the sky.  Our long-term goals are
\textsl{(1)}~to relieve the astronomical community of the onerous task
of solving for precise WCS, \textsl{(2)}~to locate, recover, and
repair badly archived or corrupted data, and \textsl{(3)}~to propagate
standards for the recording and transmission of WCS information.

In this first stage, we demonstrate that we can determine the WCS for
public Sloan Digital Sky Survey (SDSS) images using only the $(x,y)$
positions in the image of compact sources, even when we have \emph{no
first guess} or prior information of any kind as to pointing,
rotation, or plate scale (although in fact the configurations we
consider do not span much range in scale).  Our system works by
matching quadrangles of compact sources in the \usnob\ astrometric
catalog to quadrangles of compact sources in the input image.  We
employ tree methods for fast searching.  No solutions (\ie,
determinations of pointing, orientation, and scale) are considered
``good'' unless they lead to successful, precise WCS parameter fits
and pass a strong statistical test designed to flag false positives.
Our system gets good solutions for [DWH:??]~percent of SDSS images,
and the time to go from the $x,y$ list of source locations in the
input image to pointing, rotation, and scale ranges from [DWH:??]  to
[DWH:??]~s.  By taking sub-frames of SDSS images, we show that we can
reliably get good solutions for images as small as [DWH??].

\end{abstract}

\keywords{
  methods: statistical
}

\section{Introduction}

We are entering the brave new world of enormous, heterogeneous,
distributed astronomical data sets, from many large surveys and many
data archives, including large-angle homogeneous surveys such as SDSS,
2MASS, and GALEX, and pointed observatories with large public data
archives, such as HST, Spitzer, Chandra, and NOAO.  Straightforward
interoperability of these data (\eg, comparisons, matching,
variability and proper-motion studies) requires that they all have
correct, standards-compliant astrometric (WCS) information.  The data
are not always so supplied; the community needs an automated
astrometry system that will provide.

In addition, for individudal investigators using data from
observatories, determination and storage of WCS information can be a
significant obstacle to science.  For example, even well-understood
HST images have systematic astrometric offsets (relative to the
USNO-B1.0 system, for example), and every HST image user must re-solve
the astrometry locally to do precise source matching.  There are much
more serious issues with ground-based telescope data, where the
observatory-installed WCS headers are approximate and often grossly
wrong.  In general, many person-hours go into the construction of the
WCS for a typical observing run, and in the end, the investigators
often do not encode their results in a standards-compliant way or in a
community-accessible repository, so later users (re-users) must spend
those person-hours all over again.  An automated system available to
all investigators would save the community enormous amounts of
research time, and remove a significant barrier to science.

Finally, there are many legacy data sets, such as photographic plate
archives, that are scanned (digitized), or will be scanned in the near
future.  Once in electronic form, these data have enormously increased
value for astronomy in the time-domain.  Oddly, for many
organizations, the scanning of the plate material is less onerous than
the manual data entry of pointing and rotation.  An automated system
obviates this need.

For these reasons we have begun the project of making an ``astrometry
engine'' that determines pointing, rotation, and scale for any input
image, finds best-fit values for WCS parameters, and stores that WCS
solution in the image header in a standards-compliant form.  In
principle, we would like this system to work even in cases in which
the user does not know anything about the pointing, rotation or scale
of the input image, such as when plates are newly scanned, when images
have been badly archived, or when instruments have output corrupted or
incorrect image headers.

There are several technical challenges for this project, but perhaps
the greatest is the ``blind astrometry'' problem: to determine an
image's pointing, rotation, and scale with no prior guess whatsoever.
Indeed, it is not obviously possible.  However, in a pilot project,
blind astrometry has been shown to be theoretically possible for at
least some kinds of astronomical images \citep{harvey04a}.  In cases
where scale \emph{is} known, and (importantly) the images span very
large angles (tens of degrees), it is easily solved; this is the
``lost in space'' problem, solved for satellites that need to
re-determine their orientation using images taken of star fields after
a loss of control [DWH: citation].  It has also been shown that it is
possible to determine the \emph{relative} pointing, rotation, and
scale of two images that are known to overlap, even if nothing else is
known in advance \citep{groth86a}; indeed this latter problem is the
blind astrometry problem on a small scale and its solution bears some
resemblance to the solution to the much harder problem presented
here.

In this work we use SDSS images to show that the hardest form of the
blind astrometry problem---no prior information at all about image
pointing, rotation, or scale---is solvable for real, science-grade
astronomical images of varying sizes, bandpasses, and depths.

\section{Method}

[DWH: should this paragraph be in the introduction?]  In what follows,
we will describe how we determine the pointing, rotation, and scale of
an \textit{input image}, in the astrometric system defined by a
standard \textit{catalog} (here the \usnob\ catalog [DWH:
reference??]), by comparing the configurations of compact sources in
the input image to configurations in the catalog.  For the purposes of
what follows, we refer to any compact source as a ``\textit{star}'',
for the reason that in fact the astrometrically measured sources in
deep astrometric catalogs (in particular in the the \usnob\ catalog)
are not overwhelmingly stars.

The basic method we employ is to locate stars in the input image, and
look up sets of four (\textit{quadrangles}) in an \textit{index} of
quadrangles created from stars in the catalog.

[DWH: why quadrangles and not triangles or pentagons??]

\subsection{Stellar quadrangles}

The goal of the project is to determine---without prior
knowledge---the pointing, rotation, and scale of the input image, so
we desire a geometric description of a quadrangle of stars that does
not depend on any of these.  Additionally, to simplify the indexing
and search of the quadrangle database we construct, we desire the
parameters of the geometric description to be uniformly distributed in
a compact volume of parameter space.  The parameterization of
quadrangles we settled on is the following:

Choose four stars (all nearby on the sky) from the catalog.  The most
widely separated pair is labeled $\AAA$ and $\BBB$ (with the choice of
which is $\AAA$ arbitrary) and the two remaining stars are labeled
$\CCC$ and $\DDD$ (again with the choice arbitrary).  Use the
mid-point between $\AAA$ and $\BBB$ as a tangent point and project all
four stars to the tangent plane so defined using the tangent
(pinhole-camera) projection.  On the tangent plane, make star $\AAA$
the point $(\AAA_u,\AAA_v)=(0,0)$ and star $\BBB$ the point
$(\BBB_u,\BBB_v)=(1,1)$ in a right-handed orthonormal coordinate
system $(u,v)$.  The parameterization of the quadrangle is then the
$(u,v)$ coordinates of stars $\CCC$ and $\DDD$, or
$(\CCC_u,\CCC_v,\DDD_u,\DDD_v)$.

In addition to the above description, we require that all indexed
quadrangles have all four coordinates $(\CCC_u,\CCC_v,\DDD_u,\DDD_v)$
each on the interval $[0,1]$ (\ie, we require that the stars $\CCC$
and $\DDD$ lie in the square box of which $\AAA$ and $\BBB$ define
opposite corners.  Figure~\ref{fig:quad} shows this parameterization
for sixteen legal quadrangles, randomly produced.

Stellar quads parameterized this way have two ambiguities:  Stars
$\AAA$ and $\BBB$ can be swapped; this effectively rotates the $u,v$
coordinate system by $\pi$, or it transforms
\begin{equation}
(\CCC_u,\CCC_v,\DDD_u,\DDD_v) \stackrel{\AAA\BBB}{\rightarrow}
  (1-\CCC_u,1-\CCC_v,1-\DDD_u,1-\DDD_v) \quad ;
\end{equation}
and stars $\CCC$ and $\DDD$ can be swapped; this exchanges $\CCC$ and
$\DDD$ in the parameter space or it transforms
\begin{equation}
(\CCC_u,\CCC_v,\DDD_u,\DDD_v) \stackrel{\CCC\DDD}{\rightarrow}
  (\DDD_u,\DDD_v,\CCC_u,\CCC_v,) \quad .
\end{equation}
There is also a third possible ambiguity: The image being solved can
have a parity opposite to that assumed.  This parity swap is an
ambiguity not for each individual quadrangle, but for the coordinate
system of all indexed quadrangles (or all quadrangles in the input
image); it effectively swaps coordinates $u$ and $v$ or it transforms
\begin{equation}
(\CCC_u,\CCC_v,\DDD_u,\DDD_v) \stackrel{\mathrm{parity}}{\rightarrow}
  (\CCC_v,\CCC_u,\DDD_v,\DDD_u) \quad .
\end{equation}

If stars are unclustered at small scales (this assumption is not valid
for any real catalog, though clustering is small in \usnob), and if
the quadrangles all span small angles, the chosen quads will fill the
four-dimensional space $(\CCC_u,\CCC_v,\DDD_u,\DDD_v)$ uniformly.  If
the angular number density (number per solid angle) of stars in the
catalog is $\Gamma$ and the catalog stars are uniformly distributed on
the sky (this assumption is also not valid for any real catalog), and
if quadrangles are considered only when the angle $\thetaAB$ between
stars $\AAA$ and $\BBB$ is between $\thetamin$ and $\thetamax$, then
the expectation value $\ENABCD$ of the total number $\NABCD$ of valid
stellar quadrangles in a large sky region is
\begin{equation}
\ENABCD = \frac{\pi}{48}\,\Gamma^4\,\Omega_\mathrm{total}\,
  [\thetamax^6-\thetamin^6] \quad ,
\end{equation}
where $\Omega_\mathrm{total}$ is the solid angle of the sky covered by
the catalog, $4\pi$~ster if the catalog is all-sky.  Because the
expectation $\ENABCD$ scales so strongly with the stellar density
$\Gamma$, the true number of allowed quads in any real stellar catalog
(which has variations in $\Gamma$ over the sky) will in fact be very
different from this homogeneous prediction.

\subsection{Index}

We do not index all possible quadrangles in the catalog, but rather we
make a subsample that has a limited range in $\thetaAB$, and good
coverage in parameter space and pointing on the sky.  The smallest
angular sizes $\thetaAB$ on which we can make large numbers of
quadrangles are on the order of $\thetaAB\sim\Gamma^{-(1/2)}$, or of
order arcmin with the \usnob\ catalog outside the Galactic plane.
There are of order $10^8$ patches of this size on the sky; ideally the
index ought to contain roughly this many quadrangles and fully cover
the sky.

[DWH: how do we choose quadrangles to index?]

[DWH: KD tree]

[DWH: Figure showing example catalog on sky and in code space?]

\subsection{Test}

Each parameter in the set $(\CCC_u,\CCC_v,\DDD_u,\DDD_v)$ can be
determined with an uncertainty of about $\Delta\theta/\thetaAB$, where
$\Delta\theta$ represents the combined uncertainties of the \usnob\
catalog, the stellar position determinations in the input image, and
non-square distortions in the input image.  This combined uncertainty
$\Delta\theta$ is usually of order arcsec, so for the smallest
quadrangles we can index, the uncertainty on each of the four
parameters $(\CCC_u,\CCC_v,\DDD_u,\DDD_v)$ is of order $1/60$.  The
probability that a quandrangle in the index will incorrectly match by
accident any particular quadrangle in the image is about $10^{-7}$.
If the index contains $10^8$ quadrangles, each quadrangle from the
input image will match of order 10 indexed quadrangles; finding the
image pointing, rotation, and scale is a task of finding two or more
image quadrangles that match index quadrangles that \emph{agree} on
the pointing, rotation, and scale.

[DWH: how do we make a list of compact sources in the $x,y$ coordinate
  system of the input image?]

[DWH: in what order do we loop over quadrangles?  I think we need to
  order them by brightness as reported by our star finder.]

[DWH: by what criterion do we consider a quadrangle matched?]

[DWH: how do we decide that two quadrangle matches ``agree''?]

\subsection{Tweak and check}

[DWH: how do we encode WCS?]

[DWH: how do we tweak up to a precise WCS?  To what order do we go?]

[DWH: on what basis do we decide that we tweaked to a correct WCS
  solution?]

\subsection{Limitations}

[DWH: no use of magnitudes/brightnesses except to order the stars at
  test time.]

[DWH: $\AAA$--$\BBB$ swaps, $\CCC$--$\DDD$ swaps.]

[DWH: parity]

[DWH: scalings]

\section{Results}

\section{Discussion}

\acknowledgements
It is a pleasure to thank Darcy Duke and Doug Finkbeiner\ldots

[DWH: grant numbers??]

\bibliographystyle{apj}
\bibliography{apj-jour,ccpp}

\clearpage
\begin{figure}
\plotone{quad.ps}
\caption{Sixteen example quadrangles of stars.  The stars are
  represented by black boxes labeled $\AAA$, $\BBB$, $\CCC$, and
  $\DDD$.  The coordinates of stars $\CCC$ and $\DDD$ in the $u,v$
  coordinate systems defined by stars $\AAA$ and $\BBB$ are shown
  graphically with grey lines and also given numerically for each
  quadrangle.\label{fig:quad}}
\end{figure}

\end{document}
