\documentclass[12pt,preprint]{aastex}
\newcommand{\latin}[1]{\textit{#1}}
\newcommand{\ie}{\latin{i.e.}}
\newcommand{\eg}{\latin{e.g.}}
\newcommand{\cf}{\latin{cf.}}
\newcommand{\etc}{\latin{etc.}}
\newcommand{\etal}{\latin{et~al.}}
\begin{document}

\title{
  Automated Astrometry I:\\
  Blind matching of image positions to stars on the sky.
}
\author{
  [tba]
}

\begin{abstract}
We report on the first stage in an ambitious project in ``blind
astrometry,'' \ie, the determination of the precise relationship
between $x$--$y$ positions in an image and RA--Dec positions on the
sky with no prior knowledge of the pointing, orientation, or scale of
the image.  Our long-term goals are \textsl{(1)}~to relieve the
astronomical community of the onerous task of solving for precise
``world coordinate system'' (WCS) information, and \textsl{(2)}~to
maintain standards for the recording and transmission of WCS
information for the new era of distributed, heterogeneous, and
multi-wavelength data sets.

In this first stage, we demonstrate that we can determine the
pointing, orientation, and scale of public Sloan Digital Sky Survey
(SDSS) images, and public KPNO 4-m Mosaic camera individual chip
(Mosaic) images, using only the positions of unresolved sources, with
no first guess or prior information of any kind.  Our system employs
geometric properties of quadrangles of stars in the USNO-B1.0
astrometric catalog, and methods for fast searching.  No solutions
(\ie, determinations of pointing, orientation, and scale) are
considered ``good'' unless they pass statistical tests designed to
flag false positives.  Our system gets good matches for
[DWH:??]~percent of SDSS images and [DWH:??]~percent of Mosaic images.

[DWH: \etc??]
\end{abstract}

\keywords{
  methods: statistical
}

\section{Introduction}

We are entering the brave new world of enormous, heterogeneous,
distributed astronomical data sets, from many large surveys and many
data archives, including large-angle homogeneous surveys such as SDSS,
2MASS, and GALEX, and pointed observatories with large public data
archives, such as HST, Spitzer, Chandra, and KPNO.  Straightforward
interoperability of these data (\eg, comparisons, matching,
variability and proper-motion studies) requires that they all have
correct, standards-compliant astrometric (WCS) information.  The data
are not so supplied; the community needs an automated astrometry
system that will provide.

In addition, for individudal investigators using data from
observatories, determination and storage of WCS information can be a
significant obstacle to science.  For example, even the
well-understood HST images have systematic astrometric offsets, and
every HST image user must re-solve the astrometry locally to do
precise source matching.  There are much more serious issues with
telescope data, where the observatory-installed WCS headers are
approximate and often grossly wrong.  In general, many person-hours go
into the construction of the WCS for a typical observing run, and in
the end, the investigators often do not encode their results in a
standards-compliant way or in a community-accessible repository, so
later users (re-users) must spend those person-hours all over again.
An automated system available to all investigators would save the
community enormous amounts of research time, and remove a significant
barrier to science.

Finally, there are many legacy data sets, such as photographic plate
archives, that lie in archival disarray.  These data have enormous
untapped value for astronomy in the time-domain.  Oddly, for many
organizations, the scanning (digitization) of the plate material is
less onerous than the manual data entry of pointing and rotation.  An
automated system solves this problem.

For these reasons we have begun the project of making a

\acknowledgements
It is a pleasure to thank\ldots

[DWH: grant numbers??]

\end{document}
