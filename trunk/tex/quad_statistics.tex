\documentclass[letterpaper]{article}
\newcommand{\starlabel}[1]{\mathsf{#1}}
\newcommand{\AAA}{\starlabel{A}}
\newcommand{\BBB}{\starlabel{B}}
\newcommand{\CCC}{\starlabel{C}}
\newcommand{\DDD}{\starlabel{D}}
\newcommand{\thetaAB}{\theta_{\AAA\BBB}}
\newcommand{\dtheta}{\mathrm{d}\theta}
\newcommand{\OmegaAB}{\Omega_{\AAA\BBB}}
\newcommand{\ENCD}{E(N_{\CCC\DDD})}
\newcommand{\ENB}{E(N_{\BBB})}
\newcommand{\ENBCD}{E(N_{\BBB\CCC\DDD})}
\newcommand{\ENABCD}{E(N_{\AAA\BBB\CCC\DDD})}
\newcommand{\Ntotal}{N_\mathrm{tot}}
\newcommand{\Omegatotal}{\Omega_\mathrm{tot}}
\newcommand{\thetamin}{\theta_\mathrm{min}}
\newcommand{\thetamax}{\theta_\mathrm{max}}
\begin{document}
\section*{A note on the statistics of stellar quadrangles}

Recall that we are making quadrangles $\AAA\BBB\CCC\DDD$ of stars,
where $\AAA$ and $\BBB$ are the two most widely separated, and $\CCC$
and $\DDD$ are two more stars, required to fall within the square box
of which $\AAA$ and $\BBB$ define the diagonally separated corners.

If the angular separation of $\AAA$ and $\BBB$ is $\thetaAB$, then in
the small-angle approximation, the square in which $\CCC$ and $\DDD$
must fall has solid angle
\begin{equation}
\OmegaAB = \frac{1}{2}\,\thetaAB^2 = Q\,\thetaAB^2 \quad ,
\end{equation}
where the factor $1/2$ has been symbolized ``$Q$'' in case we switch
to a different-shaped restriction on $\CCC$ and $\DDD$.  If the
density of stars on the sky (number per solid angle) is $\Gamma$, and
we assume that \emph{there is no angular clustering of stars}, then
the mean number of stars $\nu$ in this angular patch is
$\nu=\Gamma\,\OmegaAB$, and the actual number $n$ of stars in the
patch is (again assuming no clustering) drawn from the Poisson
distribution, to wit
\begin{equation}
p(n) = \frac{\nu^n\,e^{\nu}}{n!} \quad .
\end{equation}
We can only make a quad if $n\geq2$, in which case in fact we have
${n\choose 2}$ unique choices for $\CCC$ and $\DDD$.

Once $\AAA$ and $\BBB$ have been chosen, the expectation value $\ENCD$
for the number of unique quads (choices of $\CCC$ and $\DDD$) is found
by summing over all values of $n$, or
\begin{eqnarray}\displaystyle
\ENCD & = & \sum_{n=2}^{\infty} {n\choose 2}\,
              \frac{\nu^n\,e^{\nu}}{n!} \\
      & = & \frac{1}{2}\,\nu^2 \sum_{n=0}^{\infty} 
              \frac{\nu^n\,e^{\nu}}{n!} \\
      & = & \frac{1}{2}\,\nu^2 \\
      & = & \frac{1}{2}\,Q^2\,\Gamma^2\,\thetaAB^4 \quad ,
\end{eqnarray}
where the sum from 2 to $\infty$ is converted to a sum from 0 to
$\infty$ by dividing out a $\nu^2$ and some combinatoric factors.  I
(DWH) feel certain that we could have guessed this result up-front
with some clever argument.

We now know how many $\CCC\DDD$ pairs there are, given an $\AAA\BBB$
pair.  How many $\AAA\BBB$ pairs are there?  For each choice of $\AAA$
(\emph{again} assuming no clustering), the expectation value $\ENB$ of
the number of choices of $\BBB$ in an annulus of radius $\thetaAB$ and
width $\dtheta$ is (again in the small-angle approximation)
\begin{equation}
\ENB = 2\pi\,\Gamma\,\thetaAB\,\dtheta \quad .
\end{equation}
The expectation value $\ENBCD$ of the number of quads that can be made
with this choice of $\AAA$ at this angular separation is then
\begin{equation}
\ENBCD = \pi\,Q^2\,\Gamma^3\,\thetaAB^5\,\dtheta \quad .
\end{equation}

The total number of choices for $\AAA$ is the density $\Gamma$ times
the total survey solid angle $\Omegatotal$; the total number of unique
quads is the product of this and \emph{half} of $\ENBCD$ (because each
$\AAA\BBB$ choice can be swapped).  So the expectation value $\ENABCD$
for the total number of quads is
\begin{eqnarray}\displaystyle
\ENABCD & = & \frac{\pi\,Q^2}{2}\,\Omegatotal\,\Gamma^4\,
                \int_{\thetamin}^{\thetamax}\,\theta^5\,\dtheta \\
\ENABCD & = & \frac{\pi\,Q^2}{2}\,\frac{\Ntotal^4}{\Omegatotal^3}\,
                \int_{\thetamin}^{\thetamax}\,\theta^5\,\dtheta \quad ,
\end{eqnarray}
where the first form is in terms of the angular number density
$\Gamma$ and the second is in terms of the total number of stars
$\Ntotal$, and where $\AAA\BBB$ pairs are allowed to come from any
angular separation from $\thetamin$ to $\thetamax$.  In both of these
expressions we have implicitly assumed that the angular density of
stars is constant on the sky.

\end{document}
